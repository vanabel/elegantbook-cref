% !Mode:: "TeX:UTF-8"
\documentclass[cn,scheme=chinese,mode=fancy]{elegantbook} 

\usepackage[nameinlink,capitalize]{cleveref}

% 配置图表的 cref 名称
\crefname{figure}{图}{图}
\crefname{table}{表}{表}

% 在 \appendix 命令执行前,重新配置 chapter 的引用
% 使用 \crefalias 将 chapter 别名为 appchapter
\usepackage{etoolbox}
\pretocmd{\appendix}{%
  \crefalias{chapter}{appchapter}%
}{}{}

% 配置正文章节的引用格式和名称
\crefname{chapter}{章}{章}
\crefformat{chapter}{#2第\zhnumber{#1}章#3}
\crefmultiformat{chapter}{#2第\zhnumber{#1}章#3}{以及#2第\zhnumber{#1}章#3}{、#2第\zhnumber{#1}章#3}{以及#2第\zhnumber{#1}章#3}

\crefformat{section}{#2第#1节#3}

% 配置附录的引用格式和名称
\crefname{appchapter}{附录}{附录}
\crefformat{appchapter}{#2附录~#1#3}
\crefmultiformat{appchapter}{#2附录~#1#3}{以及#2附录~#1#3}{、#2附录~#1#3}{以及#2附录~#1#3}

\title{附录引用示例}
\subtitle{使用 Cleveref 正确引用附录}
\author{ElegantBook Cleveref 增强版}
\date{\today}

\begin{document}

\maketitle

\tableofcontents

\chapter{引言}\label{chap:intro}

本文档展示如何使用 \verb|cleveref| 宏包正确配置附录的引用格式。

\section{问题背景}\label{sec:background}

在使用 \verb|cleveref| 时,附录(appendix)和正文章节(chapter)使用相同的计数器,导致引用格式混乱。

\section{解决方案}\label{sec:solution}

通过以下配置,可以实现:
\begin{itemize}
\item 正文章节引用:第一章、第二章
\item 附录引用:附录 A、附录 B
\end{itemize}

关键代码:
\begin{verbatim}
\pretocmd{\appendix}{%
  \crefalias{chapter}{appchapter}%
}{}{}
\crefformat{chapter}{#2第\zhnumber{#1}章#3}
\crefformat{appchapter}{#2附录~#1#3}
\end{verbatim}

\chapter{第二章内容}\label{chap:second}

这是第二章的内容。我们可以引用:
\begin{itemize}
\item \cref{chap:intro}:引用第一章
\item \cref{sec:background}:引用章节
\item \cref{chap:second}:自引用
\end{itemize}

\section{定理示例}\label{sec:theorem}

\begin{theorem}{勾股定理}{pyth}
直角三角形的两条直角边的平方和等于斜边的平方。
\end{theorem}

\begin{definition}{素数}{prime}
大于 1 的自然数中,除了 1 和它本身外,不能被其他自然数整除的数。
\end{definition}

引用定理和定义:
\begin{itemize}
\item \cref{thm:pyth}:引用定理
\item \cref{def:prime}:引用定义
\item \cref{thm:pyth,def:prime}:多个引用
\end{itemize}

\chapter{第三章}\label{chap:third}

在进入附录之前,让我们再次测试引用:
\begin{itemize}
\item 引用前面的章节:\cref{chap:intro,chap:second,chap:third}
\item 引用小节:\cref{sec:background,sec:solution,sec:theorem}
\end{itemize}

注意:\textbf{下面进入附录部分},引用格式将会改变!

% ==================== 附录部分 ====================
\appendix

\chapter{附录示例 A}\label{appx:A}

这是附录 A 的内容。

\section{附录中的小节}\label{appx:A:sec1}

附录中也可以有小节。

\subsection{附录中的子小节}

更细的层级。

\chapter{附录示例 B}\label{appx:B}

这是附录 B 的内容。

\section{引用测试}\label{appx:B:test}

现在测试各种引用:

\subsection{引用正文章节}

\begin{itemize}
\item \cref{chap:intro}:引用第一章(正文)
\item \cref{chap:second}:引用第二章(正文)
\item \cref{chap:third}:引用第三章(正文)
\item \cref{sec:background}:引用正文中的小节
\end{itemize}

\textbf{预期效果}:这些引用应该显示为"第一章"、"第二章"等。

\subsection{引用附录}

\begin{itemize}
\item \cref{appx:A}:引用附录 A
\item \cref{appx:B}:自引用附录 B
\item \cref{appx:A:sec1}:引用附录中的小节
\item \cref{appx:A,appx:B}:引用多个附录
\end{itemize}

\textbf{预期效果}:这些引用应该显示为"附录 A"、"附录 B"等。

\subsection{混合引用}

\begin{itemize}
\item \cref{chap:intro,appx:A}:混合引用正文和附录
\item \cref{chap:second,chap:third,appx:A,appx:B}:多个混合引用
\end{itemize}

\chapter{附录示例 C:数学内容}\label{appx:C}

\begin{theorem}{费马大定理}{fermat-app}
当 $n > 2$ 时,方程 $x^n + y^n = z^n$ 没有正整数解。
\end{theorem}

\begin{proof}
证明由 Andrew Wiles 于 1995 年完成。
\end{proof}

引用附录中的定理:\cref{thm:fermat-app}

同时引用正文和附录中的定理:\cref{thm:pyth,thm:fermat-app}

\section{总结}

通过正确配置 \verb|cleveref|,我们实现了:
\begin{enumerate}
\item 正文章节使用中文数字编号
\item 附录使用英文字母编号
\item 引用时自动显示正确的格式
\item 支持混合引用正文和附录
\end{enumerate}

\end{document}

