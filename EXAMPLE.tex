% !TeX program = xelatex
% ElegantBook with Cleveref - 快速入门示例

\documentclass[cn,scheme=chinese,mode=fancy]{elegantbook}
\usepackage[nameinlink,capitalize]{cleveref}

\title{ElegantBook 与 Cleveref}
\subtitle{快速入门示例}
\author{作者}
\date{\today}

\begin{document}

\maketitle
\tableofcontents

\chapter{基本用法}\label{chap:basic}

\section{定理环境}\label{sec:theorem}

\begin{theorem}{勾股定理}{pythagoras}
在直角三角形中,直角边的平方和等于斜边的平方:
\[ a^2 + b^2 = c^2 \]
\end{theorem}

\begin{definition}{素数}{prime}
大于 1 的自然数,除了 1 和它本身外,不能被其他自然数整除。
\end{definition}

\begin{lemma}{辅助引理}{helper}
这是一个辅助引理的内容。
\end{lemma}

\begin{proposition}{基本命题}{basic}
这是一个基本命题的内容。
\end{proposition}

\begin{corollary}{推论}{corollary}
从上述定理可以得出这个推论。
\end{corollary}

注意:上述环境自动连续编号(1.1, 1.2, 1.3, 1.4, 1.5)

\section{引用示例}\label{sec:ref}

\subsection{单个引用}

\begin{itemize}
\item 章引用:参见\cref{chap:basic}
\item 节引用:详见\cref{sec:theorem}
\item 定理引用:根据\cref{thm:pythagoras}
\item 定义引用:由\cref{def:prime}可知
\end{itemize}

\subsection{多重引用}

\begin{itemize}
\item 多个定理:\cref{thm:pythagoras,lem:helper,prop:basic}
\item 章节混合:\cref{chap:basic,sec:theorem,sec:ref}
\end{itemize}

\section{图表和公式}

\begin{figure}[htbp]
\centering
\rule{5cm}{3cm}  % 黑色方块代替实际图片
\caption{示例图片}\label{fig:example}
\end{figure}

\begin{table}[htbp]
\centering
\caption{示例表格}\label{tab:example}
\begin{tabular}{ccc}
\hline
列1 & 列2 & 列3 \\
\hline
数据1 & 数据2 & 数据3 \\
\hline
\end{tabular}
\end{table}

Einstein 质能方程:
\begin{equation}\label{eq:einstein}
E = mc^2
\end{equation}

引用示例:
\begin{itemize}
\item 图引用:如\cref{fig:example}所示
\item 表引用:见\cref{tab:example}
\item 公式引用:由\cref{eq:einstein}可知
\end{itemize}

\chapter{高级用法}\label{chap:advanced}

\section{习题环境}

\begin{problemset}

\item 求解方程 $x^2 - 3x + 2 = 0$

\begin{solution}
因式分解:$(x-1)(x-2) = 0$,所以 $x = 1$ 或 $x = 2$。
\end{solution}

\item 证明 $\sqrt{2}$ 是无理数

\begin{solution}
使用反证法。假设 $\sqrt{2}$ 是有理数...
\end{solution}

\item 计算积分
\begin{equation}
\int_0^1 x^2 dx
\end{equation}

\begin{solution}
$$\int_0^1 x^2 dx = \left[\frac{x^3}{3}\right]_0^1 = \frac{1}{3}$$
\end{solution}

\end{problemset}

\textbf{提示}:使用 \texttt{noanswer} 选项可以隐藏答案:
\begin{verbatim}
\documentclass[cn,mode=fancy,noanswer]{elegantbook}
\end{verbatim}

\section{自定义定理环境}

可以使用 \verb|\elegantnewtheorem| 创建自定义环境:

\begin{verbatim}
\elegantnewtheorem{conjecture}{猜想}{prostyle}{conj}[theorem]
\end{verbatim}

\section{章节引用测试}

跨章节引用:\cref{chap:basic,chap:advanced}

连续章节范围:当引用多个连续章节时,会自动合并为范围格式。

\appendix

\chapter{附录 A}\label{appx:A}

这是附录内容。

要实现附录引用格式,需要在导言区添加:
\begin{verbatim}
\usepackage{etoolbox}
\crefname{appchapter}{附录}{附录}
\crefformat{appchapter}{#2附录~#1#3}
\pretocmd{\appendix}{\crefalias{chapter}{appchapter}}{}{}
\end{verbatim}

\chapter{总结}

\section{主要特性}

\begin{enumerate}
\item 定理环境自动连续编号
\item Cleveref 引用格式自动配置
\item 中文连接词和数字格式
\item Problemset 环境增强
\item 极简的导言区配置
\end{enumerate}

\section{文档类选项}

\begin{itemize}
\item \texttt{chapter}(默认):定理按章编号
\item \texttt{section}:定理按节编号
\item \texttt{noanswer}:隐藏 problemset 中的答案
\end{itemize}

\end{document}
