% !Mode:: "TeX:UTF-8"
% !Mode:: "TeX:UTF-8"
\documentclass[cn,scheme=chinese, mode=fancy]{elegantbook} 

\usepackage[nameinlink,capitalize]{cleveref}
\elegantnewtheorem{thm}{定理}{thmstyle}{thm}[theorem]
\elegantnewtheorem{lem}{引理}{thmstyle}{lem}[theorem]
\elegantnewtheorem{defn}{定义}{defstyle}{defn}[theorem]
\elegantnewtheorem{cor}{推论}{thmstyle}{cor}[theorem]
\elegantnewtheorem{prop}{命题}{prostyle}{prop}[theorem]
\elegantnewtheorem{rmk}{注意}{attstyle}{rmk}[theorem]

\title{测试定理环境统一编号}
\subtitle{Test Continuous Numbering}
\author{测试作者}
\date{\today}

\begin{document}

\maketitle

\chapter{测试章节}

\section{测试节}

\begin{thm}{}{thm1}
这是一个定理,应该编号为 1.1。
\end{thm}

\begin{definition}{}{def1}
这是一个定义,应该编号为 1.2(与定理连续编号)。
\end{definition}

\begin{lemma}{}{lem1}
这是一个引理,应该编号为 1.3。
\end{lemma}

\begin{proposition}{}{prop1}
这是一个命题,应该编号为 1.4。
\end{proposition}

\begin{corollary}{}{cor1}
这是一个推论,应该编号为 1.5。
\end{corollary}

\begin{axiom}{}{axi1}
这是一个公理,应该编号为 1.6。
\end{axiom}

\begin{postulate}{}{pos1}
这是一个公设,应该编号为 1.7。
\end{postulate}

\section{引用测试}

引用定理:\cref{thm:thm1}

引用定义:\cref{def:def1}

引用引理:\cref{lem:lem1}

引用命题:\cref{pro:prop1}

引用推论:\cref{cor:cor1}

引用公理:\cref{axi:axi1}

引用公设:\cref{pos:pos1}

多个引用:\cref{thm:thm1,def:def1,lem:lem1}

\chapter{第二个章节}

\section{测试可选参数语法}

下面展示通过 \verb|\elegantnewtheorem| 创建的环境支持的各种语法:

\subsection{无参数的定理}

\begin{thm}
这是一个完全没有参数的定理,内容会正确显示在这里,不会被误认为是标题参数。
编号应该是 2.1。
\end{thm}

\subsection{只有标题的定理}

\begin{thm}{重要定理}
这是一个只有标题的定理,没有标签,因此不能被引用。
编号应该是 2.2。
\end{thm}

\subsection{有标题和标签的定理}

\begin{thm}{Cauchy-Schwarz 不等式}{cs}
这是一个既有标题又有标签的定理,可以被引用。
编号应该是 2.3。
\end{thm}

现在引用这个定理:\cref{thm:cs}

\subsection{其他环境的测试}

\begin{lem}
没有参数的引理,编号应该是 2.4。
\end{lem}

\begin{defn}{拓扑空间}
只有标题的定义,编号应该是 2.5。
\end{defn}

\begin{prop}{}{basic}
没有标题但有标签的命题,编号应该是 2.6。
\end{prop}

\begin{cor}{}{important}
没有标题但有标签的推论,编号应该是 2.7。
\end{cor}

\section{综合引用测试}

引用各种环境:
\begin{itemize}
\item Cauchy-Schwarz 不等式:\cref{thm:cs}
\item 基本命题:\cref{prop:basic}
\item 重要推论:\cref{cor:important}
\item 多个引用:\cref{thm:cs,prop:basic,cor:important}
\end{itemize}

\end{document}
